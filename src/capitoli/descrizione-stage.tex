\section{Introduzione al progetto}
\intro{Storia del progetto prima del mio arrivo, azienda che ha commissionato il progetto, descrizione dello scopo della piattaforma e del suo funzionamento, motivazioni alla base della scelta di riscrittura del backend.}


\section{Requisiti}
I requisiti dello stage riportati nel piano di lavoro sono i seguenti, categorizzati per importanza:

\paragraph{Obbligatori} Requisiti primari, necessari per una buona riuscita dello stage:
\begin{itemize}
	\item gestione e pianificazione del progetto attraverso kanban board condivisa;
	\item analisi dei flussi attuali e delle API richieste;
	\item progettazione ed implementazione dei modelli e dei controller, a partire dai requisiti raccolti;
	\item analisi ed integrazione Zoom, GoToWebinar, Webex.
\end{itemize}

\paragraph{Desiderabili} Non necessari, ma che contribuiscono alla completezza del prodotto, se rispettati:
\begin{itemize}
	\item coordinamento con il cliente finale;
	\item integrazione team;
	\item integrazione stampante biglietti;
	\item suite di testing del software prodotto;
	\item documentazione completa.
\end{itemize}

\paragraph{Opzionali} Che portano del valore aggiunto al progetto:
\begin{itemize}
	\item ulteriori modifiche all'applicazione che esulano da quando riportato nel piano di lavoro.
\end{itemize}

\noindent Il conseguimento dei requisiti è stato in parte dipendente dalle decisioni di gestione del progetto da parte del \emph{project manager} e dalle richieste o dalla disponibilità del committente. In particolare è stato chiesto di dedicare una quantità limitata di tempo al testing, perché non richiesto esplicitamente e non è stato possibile implementare le integrazioni, a causa di ritardi nelle risposte del committente, necessarie per definire alcuni aspetti.


\section{Pianificazione}
\intro{Divisione settimanale del lavoro dal piano di lavoro, incluse correzioni.}




\section{Tecnologie utilizzate}
\intro{Descrizione della configurazione del framework Ruby on Rails utilizzata: librerie utilizzate, Postgres, AWS, API REST.}
