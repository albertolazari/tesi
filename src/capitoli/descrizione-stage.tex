\section{Introduzione al progetto}
\intro{Storia del progetto prima del mio arrivo, azienda che ha commissionato il progetto, descrizione dello scopo della piattaforma e del suo funzionamento, motivazioni alla base della scelta di riscrittura del backend.}


\section{Requisiti}
I requisiti dello stage riportati nel piano di lavoro sono i seguenti, categorizzati per importanza:

\paragraph{Obbligatori} Requisiti primari, necessari per una buona riuscita dello stage:
\begin{itemize}
	\item gestione e pianificazione del progetto attraverso kanban board condivisa;
	\item analisi dei flussi attuali e delle API richieste;
	\item progettazione ed implementazione dei modelli e dei controller, a partire dai requisiti raccolti;
	\item analisi ed integrazione Zoom, GoToWebinar, Webex.
\end{itemize}

\paragraph{Desiderabili} Non necessari, ma che contribuiscono alla completezza del prodotto, se rispettati:
\begin{itemize}
	\item coordinamento con il cliente finale;
	\item integrazione team;
	\item integrazione stampante biglietti;
	\item suite di testing del software prodotto;
	\item documentazione completa.
\end{itemize}

\paragraph{Opzionali} Che portano del valore aggiunto al progetto:
\begin{itemize}
	\item ulteriori modifiche all'applicazione che esulano da quando riportato nel piano di lavoro.
\end{itemize}

\noindent Il conseguimento dei requisiti è stato in parte dipendente dalle decisioni di gestione del progetto da parte del \emph{project manager} e dalle richieste o dalla disponibilità del committente. In particolare è stato chiesto di dedicare una quantità limitata di tempo al testing, perché non richiesto esplicitamente e non è stato possibile implementare le integrazioni, a causa di ritardi nelle risposte del committente, necessarie per definire alcuni aspetti.


\section{Pianificazione}
Il periodo di svolgimento dello stage era previsto tra il 26 aprile 2022 e il 1 luglio 2022, per una durata complessiva di 300 ore. Il periodo preventivato considera due settimane aggiuntive a quelle necessarie a raggiungere 300 ore, utili a coprire eventuali imprevisti. La tabella che segue mostra la pianificazione delle attività da svolgere per ogni settimana, considerando 8 ore di lavoro al giorno:

\begin{table}[h]
	\centering
	\rowcolors{2}{gray!25}{white}
	\label{tab:pianificazione}
	\begin{tabularx}{0.65 \textwidth}{c|X}
		\rowcolor{white}
		\textbf{Settimana} & \textbf{Attività} \\
		\hline
		\makecell{\textbf{1} \\ 27/04 - 29/04} & \makecell[l]{Comprensione sistema e obiettivi \\ Analisi dei requisiti} \\
		\makecell{\textbf{2} \\ 02/05 - 06/05} & Progettazione \\
		\makecell{\textbf{3} \\ 09/05 - 13/05} & \makecell[l]{Progettazione \\ Studio e setup ambiente di sviluppo} \\
		\makecell{\textbf{4} \\ 16/05 - 20/05} & Implementazione \\
		\makecell{\textbf{5} \\ 23/05 - 27/05} & Implementazione \\
		\makecell{\textbf{6} \\ 30/05 - 03/06} & Implementazione \\
		\makecell{\textbf{7} \\ 06/06 - 10/06} & \makecell[l]{Implementazione \\ Test e validazione} \\
		\makecell{\textbf{8} \\ 13/06 - 17/06} & \makecell[l]{Test e validazione \\ Documentazione}
	\end{tabularx}
	\vspace{5pt}
	\caption{Tabella della pianificazione del lavoro}
\end{table}


\intro{Descrizione della configurazione del framework Ruby on Rails utilizzata: librerie utilizzate, Postgres, AWS, API REST.}
\section{Tecnologie utilizzate}
Essendo il progetto un lavoro di riscrittura completa, è stato deciso di sfruttare quasi tutte le tecnologie comunemente utilizzate per convenzione aziendale nello sviluppo di backend:

\subsection{Ruby}
Ruby \footcite{site:ruby} è un linguaggio di \emph{scripting} interpretato, a oggetti e dinamicamente tipato, che punta alla produttività, mantenendo una sintassi semplice ed elegante e una bassa curva di apprendimento. La sintassi prevede un uso limitato della punteggiatura e di elementi considerati superflui, preferendo uno stile spesso naturale e intuitivo, che migliora la leggibilità del codice.
\begin{lstlisting}
	user.updated_at = 1.hour.ago
\end{lstlisting}
Il linguaggio è multi-paradigma, ma fortemente orientato agli oggetti: perfino i tipi primiti e i metodi vengono trattati come oggetti. 
