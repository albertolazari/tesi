\section{Introduzione}
La prima attività che ho dovuto svolgere è stata la definizione dei modelli del nuovo backend. Purtroppo l'architettura già presente nel vecchio backend, soprattutto lato database, aveva diversi difetti e andava rivista interamente, quindi è stato necessario partire da un'analisi di quanto già fatto, per capire se fosse necessario effettuare del refactor nella struttura dei dati. In molti casi le entità contenevano molti attributi superflui o non adatti ad essere associati alla specifica entità, perché più adatti ad essere associati attraverso una relazione a un'altra entità.

In questo capitolo vengono riportati i cambiamenti e le scelte più significative che sono state fatte durante questa fase, che hanno portato alla produzione del diagramma ER completo dell'applicazione, riportato nell'appendice \ref{cap:ER}.

\section{Modifiche significative effettuate}
\subsection{Utenti}
La prima tabella analizzata è stata quella degli utenti. Nella versione precedente le informazioni relative al modello degli utenti erano divise in tre entità:
\begin{itemize}
	\item \verb|e_users|: conteneva i dettagli principali degli utenti, come nome, cognome e email, i campi per la gestione dell'autenticazione e altri, che però erano inutilizzati;
	\item \verb|e_partecipants|: era dedicata agli utenti partecipanti. Conteneva un sottoinsieme dei campi di \verb|e_users|:
	\begin{itemize}
		\item \verb|companyId|;
		\item \verb|firstName|;
		\item \verb|lastName|;
		\item \verb|email|;
		\item \verb|phone|;
		\item \verb|createdAt|;
		\item \verb|status|.
	\end{itemize}
	\item \verb|e_users_levels|: memorizzava i ruoli degli utenti e, per ogni azione esposta dall'API, un campo booleano che indicava se quel ruolo avesse l'autorizzazione di effettuare quell'azione.
\end{itemize}
Ho deciso di raccogliere queste informazioni in un unico modello, con i campi dell'entità \verb|e_partecipants|, a cui ho aggiunto l'immagine dell'utente e un enumerazione che ne indica il ruolo, invece di utilizzare un'intera entità come era stato fatto precedentemente. I campi necessari per l'autenticazione sono stati omessi, perché già presenti nel \emph{template} dei progetti di Moku, che è stato usato come base di partenza anche per questo.

La motivazione alla base della scelta di non codificare le autorizzazioni all'interno di un modello è stata la complessità del processo di autorizzazione nel dominio trattato, che spesso dipende anche dall'oggetto su cui si vuole eseguire un'azione, qundi non esclusivamente dal ruolo dell'utente.

La scelta di unire i partecipanti (corretti in ``participants'', invece di ``partecipants'') al resto degli utenti è stata presa per ridurre la duplicazione dei dati, in questo caso rilevante, perché le due entità, una volta riviste, avrebbero avuto quasi tutti i campi in comune. Inoltre, seppur vero che i partecipanti non possono effettuare l'autenticazione nell'applicazione web, utilizzano l'applicazione \emph{mobile}, che utilizzerà lo stesso backend.
