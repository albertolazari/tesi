\section{Raggiungimento dei requisiti}
La tabella seguente riporta lo stato di completamento e soddisfazione dei requisiti, definiti in \S \ref{stage:requisiti}:
\begin{table}[h]
	\centering
	\rowcolors{2}{gray!25}{white}
	\label{tab:raggiungimento-obiettivi}
	\begin{tabularx}{\textwidth}{X|c|c}
		\rowcolor{white}
		\textbf{Requisito} & \textbf{Importanza} & \textbf{Stato} \\
		\hline
		\makecell[l]{Gestione e pianificazione del progetto \\ attraverso kanban board condivisa} & Obbligatorio & Rispettato \\
		\makecell[l]{Analisi dei flussi attuali e delle API \\ richieste} & Obbligatorio & Rispettato \\
		\makecell[l]{Progettazione ed implementazione dei \\ modelli e dei controller, a partire dai \\ requisiti raccolti} & Obbligatorio & Rispettato \\
		\makecell[l]{Analisi ed integrazione Zoom, \\ GoToWebinar, Webex} & Obbligatorio & Non rispettato \\
		\makecell[l]{\vspace{-6pt} \\ Coordinamento con il cliente finale \\ \vspace{-6pt}} & Desiderabile & Rispettato \\
		\makecell[l]{\vspace{-6pt} \\ Integrazione team \\ \vspace{-6pt}} & Desiderabile & Rispettato \\
		\makecell[l]{\vspace{-6pt} \\ Integrazione stampante biglietti \\ \vspace{-6pt}} & Desiderabile & Non rispettato \\
		Suite di testing del software prodotto & Desiderabile & \makecell{Non completamente \\ rispettato} \\
		\makecell[l]{\vspace{-6pt} \\ Documentazione completa \\ \vspace{-6pt}} & Desiderabile & Non rispettato \\
		\makecell[l]{Ulteriori modifiche all'applicazione che \\ esulano da quando riportato nel piano di \\ lavoro} & Opzionale & Rispettato
	\end{tabularx}
	\vspace{5pt}
	\caption{Tabella dello stato di completamento dei requisiti}
\end{table}

\noindent Come già spiegato in \S \ref{stage:requisiti}, il conseguimento dei requisiti, in alcuni casi, è stato dipendente da fattori esterni:
\begin{itemize}
	\item la suite di testing è stata iniziata, ma, dopo la richiesta da parte del \emph{Project Manager} di dedicarci una quantità di tempo limitata, l'esaustività e la completezza dei test si è ridotta, per questo il requisito è stato contrassegnato come ``Non completamente rispettato'';
	\item le comunicazioni con il committente sono state spesso soggette a rallentamenti notevoli.\ Ad esempio, durante tutta la durata dello sprint 5, l'azienda che ha commissionato il progetto si è dedicata completamente ad altri progetti, sospendendo le comunicazioni e l'avanzamento dei lavori sul progetto di Evvvents. Tra le varie domande sulle scelte da perseguire nel progetto ne erano previste alcune sullo scopo e le funzionalità che avrebbero dovuto offrire le integrazioni. Non essendo stato possibile definire questi dettagli insieme al committente, entro il periodo dello stage, le integrazioni non sono state implementate, quindi il requisito obbligatorio non è stato implementato. Nonostante questo lo stage è stato considerato concluso con successo.
\end{itemize}
Nonostante il mio stage sia giunto al termine, il progetto Evvvents rimane ancora in corso di sviluppo, con diversi aspetti da completare e implementare, data la sua grandezza e complessità. Il lavoro verrà portato avanti da un altro stagista, che prenderà il mio posto.

\section{Valutazione personale}
\intro{Messe alla prova le competenze fornite dal corso di laurea, verificata l'efficacia dei corsi e dei progetti svolti, imparato un nuovo linguaggio e framework con filosofia di sviluppo a me nuova, scoperto ambiente lavorativo aziendale con i ruoli e le dinamiche interne.}
