\section{Modelli}
La codifica dei modelli passa per tre fasi successive:
\begin{enumerate}
	\item creazione della entità del modello nel database e della classe, utilizzando le migrazioni;
	\item l'associazione del modello con altri modelli o elementi di \emph{storage};
	\item le validazioni sugli attributi e sulle associazioni dichiarate.
\end{enumerate}

\subsection{Migrazioni del database}
\intro{Comando} \texttt{rails generate} \intro{e migrazione prodotta.}

Basandosi su quanto definito nella fase di progettazione dei modelli, descritta nel capitolo \ref{cap:modelli}, questi sono stati generati utilizzando da linea di comando il generatore automatico \verb|rails generate model| o, in versione ridotta, \verb|rails g model|. Il comando accetta come argomenti:
\begin{itemize}
	\item il nome del modello, al singolare e in \emph{CamelCase};
	\item gli attributi che deve avere il modello;
	\item per ogni attributo: il suo tipo, che rispecchia, ad alto livello, i tipi comunemente disponibili per le colonne nei DBMS SQL. Normalmente è uno dei seguenti tipi nativi delle migrazioni di Rails \footcite{site:migration-types}, agnostici rispetto all'implementazione del database:
	\begin{itemize}
		\item \verb|primary_key|,
		\item \verb|string|,
		\item \verb|text|,
		\item \verb|integer|,
		\item \verb|bigint|,
		\item \verb|float|,
		\item \verb|decimal|,
		\item \verb|datetime|,
		\item \verb|timestamp|,
		\item \verb|time|,
		\item \verb|date|,
		\item \verb|binary|,
		\item \verb|blob|,
		\item \verb|boolean|,
		\item \verb|references|
	\end{itemize}
	\item per ogni attributo: l'identificatore \verb|uniq|, che imposta un indice su quella colonna del database, che ne specifica l'unicità nell'entità.
\end{itemize}
Di conseguenza, la sintassi generale è la seguente:
\begin{minted}{shell}
	rails g model ModelName attr_1:type:[uniq] attr_2:type:[uniq] ...
\end{minted}

\subsection{Associazioni a modelli e file}
\intro{Associazioni di Active Record e Active Storage.}

\subsection{Validazioni}
\intro{Validazioni sugli attributi del modello e le associazioni.}

\section{Controller}
\subsection{APIController}
\intro{Descrizione dei metodi di utilità ereditati dai controller dell'API.}

\subsection{Implementazione delle action}
\intro{Descrizione ed esempio di action tipiche dei controller.}

\section{Gestione dei permessi}
\intro{Funzionamento e uso della gemma ``Pundit'' per la gestione dei permessi relativi agli endpoint dell'API, scope e metodi relativi alle action, esempio di gestione della gerarchia che andrà rivisto}

\section{Test di unità}
\intro{Descrizione della gemma ``RSpec'', che fornisce strumenti per lo sviluppo guidato dal comportamento (behaviour-driven development), esempi di modelli testati.}
