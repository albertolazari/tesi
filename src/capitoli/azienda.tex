\section{Descrizione generale}

\begin{center}
	\includegraphics[height = 4cm]{moku-logo}
\end{center}

\noindent Moku S.r.l.\ è una start-up nata nel 2013 all'interno di un progetto supportato da H-Farm. Dopo aver abbandonato il progetto si è dedicata allo sviluppo software su commissione e consulenza, per poi allontanarsi definitivamente da H-Farm a settembre 2021, muovendo la sua sede dalla \emph{farm} a Roncade a quella attuale di Treviso.

L'azienda è in continua espansione e conta circa 20 dipendenti, la maggior parte con età inferiore ai 30 anni. Questo contribuisce a mantenere l'ambiente di lavoro stimolante e accogliente per tutti, permettendo di includere i diversi studenti che ogni anno svolgono il loro stage presso l'azienda. Per questi vengono attivate proposte di progetto per i ruoli di sviluppatore backend, sviluppatore frontend e sviluppatore mobile, all'interno di team interni che lavorano a progetti reali commissionati all'azienda.

\section{Modello di sviluppo}
I progetti di Moku seguono un modello di sviluppo \emph{agile}, con metodologie basate su \emph{Scrum} \footcite{site:scrum-guide}, un framework pensato per team di sviluppo sotware di piccole dimensioni (non più di dieci membri). La metodologia adottata prevede le sueguenti caratteristiche:
\begin{itemize}
	\item il lavoro viene suddiviso in \emph{sprint}, intervalli temporali della durata di due settimane;
	\item ogni \emph{sprint} è preceduto da una riunione di pianificazione degli obiettivi, espressi sotto forma di \emph{user stories}, che esprimono le funzionalità del software da implementare, scritte in un lunguaggio naturale dal punto di vista dell'utente;
	\item uno \emph{sprint} termina con la relativa \emph{sprint review}, una riunione con il cliente che ha l'obiettivo di mostrare l'incremento prodotto nel software, attraverso dimostrazioni del funzionamento del software stesso;
	\item il team a cui viene affidato lo sviluppo di un progetto è composto da diverse figure professionali, tra cui un \emph{project manager}, il cui compito è coordinare il lavoro tra gli altri componenti e definire le \emph{user stories} da inserire nel \emph{backlog} degli \emph{sptint}, oltre a gestire la pianificazione dello \emph{sprint} stesso;
	\item all'inizio di ogni giornata lavorativa, il team si riunisce nello \emph{stand-up meeting}, una riunione della durata di circa 15 minuti, per condividere lo stato del lavoro di ogni componente, descrivere gli obiettivi del giorno e far emergere eventuali problemi sorti durante lo sviluppo.
\end{itemize}
La metodologia adottata permette di avere una comunicazione regolare ed efficace tra il team di sviluppo e il cliente, che porta a una definizione più semplice e precisa dei requisiti che il prodotto deve rispettare e a una comprensione immediata e chiara dell'avanzamento dello sviluppo da parte del cliente, attraverso le dimostrazioni pratiche effettuate nel contesto delle \emph{sprint review}.
